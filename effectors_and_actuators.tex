\subsection{Actuation Technologies}
There are many types of actuation technologies that one might find useful when building effectors for robots. Some examples of those include:
\begin{enumerate}
    \item Electric Motors -- this is the most common form of actuators because of its affordability and ease of use. It is powered by electric current.
    \item Hydraulics -- this is based on fluid pressure, and quite powerful. However, it is large, and may encounter leaking problems.
    \item Pneumatics -- this is based on air pressure, and is similar to the functioning of hydraulic actuators. The difference here is that air is compressible.
    \item Shape-Memory Alloy (SMA) -- this particular type of allow results in changes in shape when there is a change in temperature. An example of an SMA is Nitinol.
    \item Piezoelectric materials -- these are specific types of crystals that change in length when an electric current is applied across it.
\end{enumerate}
\subsubsection{The DC Motor}
Electric motors are one of the most common form of actuators used in robotics because they are simple, inexpensive, easy to use, easily available, and also available in a great variety of shapes and packages. A special type of actuator is the servo motor. These are motors that can turn their shaft to a specific position. A servo motor consists of DC motors, gears, and control and power amplifier circuits. The input to servos consists of a series of pulses. The width of said pulses encodes information about the position that the shaft of the motor is supposed to turn to. This process is known as \emph{Pulse Width Modulation}. 

In a simple model of the DC motor, electrical power input to the motor, given by $P_e = VI$ is converted to mechanical power output by the rotation of the shaft attached to the motor, given by $P_m = T\omega$. The resistance of the motor windings can be taken to be $R$, and a back-emf of $e$ is generated when the motor is running, caused by the motor windings passing through the magnetic field of the magnet in the motor assembly.
\subsection{Degree of Freedom}
\subsection{Position and Orientation of a Rigid Body}
\subsection{Coordinate Transformation}
\subsection{Types of Joints}
\subsection{Robot Kinematics}