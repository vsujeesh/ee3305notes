\subsection{Mechanical Power in Rotational Systems}
Mechanical power, $P$ is defined as $P=\frac{E}{t}$, which is the work done per unit time. Work done, $W$, is given by $W=Fs$, which is the product of force and distance. Combining both these expressions gives us $P = \frac{Fs}{t}$. Notice that $\frac{s}{t}$ corresponds to the definition of velocity, $v$ that is created by the force. Therefore, the expression for power simplifies to $P=Fv$.