\subsection{Definitions and History of Robotics}
\begin{itemize}
\item A \emph{robot} is an \emph{autonomous system} which \emph{exists} in the physical world, can \emph{sense} its environment, and can \emph{act} on it to achieve some goals.
\item An \emph{autonomous robot} acts on the basis of its \emph{own decisions}, and is not operated by a human. This is contrasted with \emph{teleoperation}, which is the act of \emph{operating a system from afar}.
\item A robot \emph{existing in the physical world} has to obey the \emph{laws of physics}.
\item A robot has to \emph{perceive information} about its environment using \emph{sensors} if it is in the physical world. However, if the robot is simulated, all the information that would be usually perceived by sensors can be magically provided to the robot without any actual perception taking place.
\item A robot also has to \emph{respond to the perceived sensory inputs} and \emph{act accordingly to achieve its desired goal}.
\end{itemize}
\subsection{Robot Components}
The 4 main components of a robot are
\begin{enumerate}
    \item A physical body (embodiment in the physical world)
    \item Sensors (sensing and perception of environmental inputs)
    \item Effectors and Actuators (take action and respond to input)
    \item A controller (allow for autonomous decisions)
\end{enumerate}
\subsection{Applications}